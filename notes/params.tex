\documentclass[12pt]{article}
\usepackage{graphicx}
\usepackage{booktabs}
\usepackage[margin=1.0in]{geometry}
\usepackage{color}
\usepackage{tabularx}
%Times New Roman 12 pt
\usepackage[T1]{fontenc}
\usepackage{mathptmx}
\usepackage{amssymb}
\usepackage{amsmath}
\usepackage{authblk}
\usepackage[backend=bibtex,style=phys]{biblatex}



\title{}
\author[1]{Lisa E. Felberg}
\author[2]{Luis A. Ruiz Pestana} 
\author[1-4]{Teresa Head-Gordon} 

\addbibresource{references}

\affil[1]{Department of Chemical and Biomolecular Engineering, University of California Berkeley, 
Berkeley, California 94720, USA}
\affil[2]{Chemical Sciences Division, Lawrence Berkeley National Labs
Berkeley, California 94720, USA}
\affil[3]{Department of Chemistry, University of California Berkeley, 
Berkeley, California 94720, USA}
\affil[4]{Department of Bioengineering, University of California Berkeley, 
Berkeley, California 94720, USA}


\date{}                     %% if you don't need date to appear
\setcounter{Maxaffil}{0}
\renewcommand\Affilfont{\itshape\small}

\begin{document}
	\maketitle
	
\section{Abstract}


\section{Introduction}

\section{Methods}

\subsection{Models}

For this study, simulations of systems with parallel graphene plates with
separations of 12, 16, 20 and 24 \r A respectively in the x-cartesian dimension.
Simulations with solute molecules include 2 solute molecules, one for each solvent
layer.

\subsection{Force fields}

The graphene parameters used were k\textsubscript{bond} = 938 \(kCal/mol/\AA^2\)
and k\textsubscript{angle} = 126 \(kCal/mol/rad^2\) with equilibrium bond and 
angle lengths of 1.4 \r A and 120 degrees respectively \cite{Hummer2001}. The
dihedral potential is harmonic with the functional form: \(E = K [ 1 + d cos(\phi)]\),
where k = 3.15 kCal/mol, and \(d = \pm 1\) \cite{Patra2009}. The carbon partial charge 
is 0.0 e, and the 6-12 Lennard-Jones pairwise potential coefficients are \(\sigma=0.070\)
and \(r_0 = 3.55 \AA\).

The benzene parameters were taken from the OPLS-AA force field \cite{Jorgensen1996}.
The atom type #145 was used for benzene carbons and #146 was used for benzene
hydrogens. This is from the papers: # J. Am. Chem. SOC. 1990, 112, 4168-4114 \cite{Jorgensen1990}.
and J. Am. Chem. Soc. 1996, 118, 11225-11236  \cite{Jorgensen1996}.  

Nonbonded parameters
#   AN AT   CHARGE     SIGMA    EPSILON 
145 06 CA   -0.115     3.550     0.070     Benzene C - 12 site JACS,112,4768-90
146 01 HA    0.115     2.420     0.030     Benzene H - 12 site  "

For bonds and angles, the following parameters were taken from AMBER all-atom
\cite{Weiner1986} (also from http://zarbi.chem.yale.edu/doc/par_opls_aam.inp)
Which has different CA-HA, and angle params.

Bonds
CA-CA 469.       1.40         TRP,TYR,PHE
CA-HA 367.       1.080        PHE, etc. 

Angles
CA-CA-CA    63.        120.     PHE(OL)
CA-CA-HA     35.       120.


Torsions
#      v1        v2        v3       v4         notes
165   0.0       7.250     0.0       0.0        CA-CA-CA-CA                           

Improper
improper_coeff 1 1.1 -1 1  # CA-CA-CA-HA for benzene 

The water model chosen is TIP4P-Ew \cite{Horn2004} was used to represent explicit 
solvent.

\subsection{Simulation protocols}

The system was built with force field parameters described, and the density was
chosen with %%%%%%%%%%%%%%%.
The simulation was perfomed with the LAMMPS software package \cite{Plimpton1995}. 
Equilibration of initial configurations were performed as follows. Systems were run in 
the NVE ensemble with a Langevin Thermostat at 298 Kelvin
for 10 ps. The water was then fixed with the SHAKE protocol \cite{Andersen1983} with a tolerance
of 10\textsuperscript{-5}, and the system was run for 10 ps in the NVT ensemble.
Finally, data was saved for analysis in a production run in the NpT ensemble,
during which all cartesian coordinates were stored every 500 timesteps. 
Systems were run in the NpT ensemble for 10 ns. Each system was run with 4 different 
replicas, initialized from identical configurations, but velocities were initialized with
different random seeds. Reported values were averaged across the ensemble.  The 
pressure in the direction orthogonal to the graphene walls was maintained at 1 atm.
Thermal control was implemented using a Nos\' e-Hoover extended Lagrangian procedure \cite{Martyna1994},
and the dynamical integration scheme was velocity-Verlet \cite{Swope1982}.
For systems containing only graphene and water, the dynamics timestep was set at 2.0 
femtoseconds. For simulations containing solute molecules, the timestep was set
at 1.0 fs. 

\subsection{Analysis}


